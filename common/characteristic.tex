
{\actuality}
Индустрия создания программного обеспечения (далее - ПО) растет очень быстрыми темпами. Согласно отчету компании Gartner, в 2022 году глобальный рынок ПО оценивался в 4.534 милли­ арда долларов США, что на 3\% больше, чем в 2021 году. Ожидается, что рост индустрии будет продолжаться и в ближайшие годы. Рост инду­стрии ПО обусловлен многими факторами. Прежде всего, сегодня ПО используется в различных отраслях, от банковского и финансового секторов до здравоохранения и государственного управления. При этом появляются новые технологии и требования, что приводит к со­зданию новых программных продуктов и услуг. Также следует отметить, что развитие интернета, мобильных устройств и облачных технологий создает новые возможности для создания и использования ПО. Более того, в условиях пандемии COVID-19 большое количество людей перешло на удаленную работу, что привело к росту спроса на ПО для удаленной работы. Развитие индустрии создания ПО требует повышения эффективность разработки новых программных продуктов и при этом предъявляет всё большие требования к их безопасности и качеству.

В настоящее время проблема обеспечения информационной безопасности в государственных и коммерческих автоматизированных системах обработки и управления информацией встаёт особенно остро. Среди основных причин этого: политическая нестабильность и стремительно возрастающий объем информатизации в мире, а также совершенствование и повсеместная популяризация компьютерных технологий, в том числе приводящая к существенному росту числа и навыков потенциальных нарушителей – формированию «армии» хакеров, зачастую действующих из коростных побуждений, в том числе в рамках кибер-операций, проводимых в интересах различных государств. Экспертные оценки прогнозируют трехкратный рост финансовых потерь от киберпреступности в ближайшие пять лет , при этом угрозы устойчивости государственных систем не поддаются исчислению, поскольку несут в себе риски функционирования государства как целостной сущности.
Современные программные системы по оценке например НИУ ВШЭ «в 98\% случаев содержат в себе open source, или открытое (свободное) ПО. При этом программы в среднем на 75\% состоят из open source компонентов». Существующее обывательское представление от том, что свободное ПО является по определению более доверенным, а следовательно – безопасным в использовании, в общем случае не является верным. Действительно, обнаружение потенциальных вредоносных «закладок» и недекларированных возможностей в проприетарном программном обеспечении, поставляемом пользователю в бинарном и/или обмундированном виде, представляет собой проблему, которая по определению отсутствует при использовании открытого ПО. Однако, вместе с тем, проприетарное ПО как правило является источником дохода разрабатывающей его компании, а следовательно существует организация (как минимум одна), непосредственно заинтересованная в положительной репутации ПО, а следовательно – отсутствии в нём публично известных (как минимум – публично-обнаруживаемых) НДВ, а также минимизации числа различных уязвимостей (архитектурных, конфигурационных, кодовых, вносимых сборочной системой).

В программировании программные библиотеки позволяют повторно использовать код, уменьшать объем работы разработчиков и сокращать время разработки программных продуктов. Программная библиотека (далее библиотека) представляет собой набор предварительно написанного кода, который можно использовать для добавления определенных функций в программное приложение. Она является своего рода строительным блоком для разработки  ПО, предоставляя разработчикам возможность повторно использовать код и экономить время.
Библиотеки могут включать в себя широкий спектр сущностей: функции, классы, структуры, перечисления и т. д.. Как правильно они разрабатываются так, чтобы их было легко использовать и интегрировать в существующее приложение, что позволяет разработчикам быстро и эффективно добавлять новые функции и возможности.
Использование программных библиотек может значительно улучшить процесс разработки за счет уменьшения объема кода, который необходимо писать с нуля. Это может привести к сокращению времени разработки, улучшению качества кода и снижению затрат на разработку. Кроме того, библиотеки могут быть общими и использоваться несколькими разработчиками и командами, что способствует сотрудничеству и обмену передовым опытом в сообществе разработчиков ПО. Это может привести к созданию высококачественных, хорошо документированных библиотек, которые можно использовать во многих различных проектах.
Стоит отметить, что существует также множество библиотек с открытым исходным кодом, которые можно использовать бесплатно. Эти библиотеки поддерживаются сообществом разработчиков и и являются ценным ресурсом для разработчиков ПО, желающих добавить новые функции в свои приложения. 

Для обнаружения ошибок и дефектов в коде программных библиотеках обычно применяются методы статического и динамического анализ. 

Статический анализ кода - это процесс анализа исходного кода без его выполнения, при котором проверятся наличие синтаксических ошибок, возможных дефектов и неправильных практик. Статический анализ кода может производиться как вручную, так и с использованием специальных инструментов, которые автоматически анализируют код на предмет соответствия стандартам и правилам программирования. Преимуществом статического анализа является его быстрота и возможность обнаружения ошибок на ранней стадии разработки. Наиболее массовый тип инструментов статического анализа выполняет поиск ошибок и дефектов в Абстрактном синтаксическом дереве (далее - АСД) - это промежуточного представления кода, которое создается компилятором в процессе его работы.

Динамический анализ кода - это процесс анализа исполняемого кода, при котором ПО запускается и тестируется в реальном времени. Динамический анализ кода может включать в себя тестирование на утечки памяти, проверку производительности, тестирование функциональности и другие виды тестирования. Динамический анализ кода может производиться как вручную, так и автоматически с использованием специальных инструментов. Преимуществом динамического анализа является возможность обнаружения проблем, которые не могут быть выявлены статическим анализом. 

Технология фаззинга является одним из поаулярных и эффективных инструментов  динамического анализа. Фаззинг позволяет обнаруживать ошибки, которые могут привести к краху программы, утечке памяти, доступу к конфиденциальной информации и другим проблемам безопасности. В последние годы популярность фаззинга как вида тестирования постоянно возрастает, и это связано с несколькими факторами. Во-первых, современные фаззеры (например, AFL, LibFuzzer) стали более интеллектуальными и способными генерировать более сложные и реалистичные входные данные, что повышает эффективность тестирования. Во-вторых, с появлением искусственного интеллекта и машинного обучения в фаззинге появилась возможность автоматически определять наиболее перспективные тестовые кейсы для дальнейшего тестирования, что значительно ускоряет процесс. Однако проблема автоматического тестирования программных библиотек ~\autocite{Shamshiri2018HowDA} до сих пор актуальна из за перечня следующих проблем:
\begin{itemize}
    \item необъявленные исключения~\autocite{Csallner2004JCrasherAA};
    \item нарушение кодовых контрактов или контекстов использования~\autocite{4222570};
    \item экспоненциальный рост объема и сложности программного кода.
\end{itemize}

{\aim} данной работы является разработка нового метода автоматизированного тестирования программных библиотек посредством автоматической генерации фаззинг-оберток для функций библиотеки на языках Си и Си++. В качестве  фаззера используются популярные платформы libFuzzer~\autocite{libFuzzer} и AFLplusplus~\autocite{AFLplusplus}.

Для~достижения поставленной цели необходимо было решить следующие {\tasks}:
\begin{enumerate}[beginpenalty=10000] % https://tex.stackexchange.com/a/476052/104425
  \item Автоматически анализировать библиотеки чтобы вывести характеристики, свойства и взаимосвязи между сущностями кода (пользовательские типы, функции, классы, структуры и т. д.). Эта информация необходима для оформления вызовы функции, метода библиотеки.
  \item Исследовать метод передачи однопоточного буфера данных фаззера для аргументов вызванной функции в фаззинг-обертке.
  \item Разработать метод автоматической генерации фаззинг-оберток для функций в условиях отсутствия контекста использования.
  \item Разработать метод автоматической генерации фаззинг-оберток для функций c учетом контекста использования в других программах.
  \item Разработать программный продукт для автоматического анализа код библиотеки, автоматической генерации фаззинг-оберток для функций и методов в библиотеке, автоматического запуска и сбора результат тестирования.
\end{enumerate}

{\compliance} Цели и задачи диссертационного исследования соответствуют направлениям исследований, предусмотрен­ ным паспортом специальности 2.3.5 «Математическое и программное обеспечение вычислительных систем, комплексов и компьютерных сетей». Область исследования настоящей диссертации включает в себя методы и алгоритмы анализа программ (пункт 1 паспорта специальности), языки программирования  и семантику программ (пункт 2), методы, архитектуры, алгоритмы, языки и программные инструменты организации взаимодействия программ и программных систем (пункт 3).

{\novelty}
\begin{enumerate}[beginpenalty=10000] % https://tex.stackexchange.com/a/476052/104425
  \item Впервые был разработан метод автоматической генерации фаззинг-оберток для функций библиотеки в условиях отсутствия контекста использования.
  \item Впервые был разработан метод полностью автоматического тестирования программных библиотек с применением статического и динамического анализа.
\end{enumerate}

{\influence} Практическая значимость заключается в том, что разработанные методы позволяют  генерировать правильные контекст вызовов, что позволяет генерировать  и компилировать фаззинг-обертки, в результате фаззинга были  обнаружены новые программные дефекты.

{\methods} В процессе решения задачи автоматического анализа автор опирается на теории языка программирования. При решении задачи автоматической генерации фаззинг-оберток для функций в библиотеке используются теория компиляторов, в том числе граф потока управления и анализ потока данных.

{\defpositions}
\begin{enumerate}[beginpenalty=10000] % https://tex.stackexchange.com/a/476052/104425
  \item Предложен метод автоматического анализа кода программного продукта.
  \item Предложен метод автоматической генерации фаззинг-оберток для функции библиотек.
  \item Разработано программное обеспечение Futag для автоматической генерации фаззинг-оберток для программных библиотек.
\end{enumerate}

{\reliability} полученных результатов обеспечивается \ldots \ Результаты находятся в соответствии с результатами, полученными другими авторами.

{\probation}
Основные результаты работы докладывались~на:
\begin{enumerate}[beginpenalty=10000] % https://tex.stackexchange.com/a/476052/104425
  \item Открытая конференция ИСП РАН им. В.П. Иванникова. Москва. 2020.
  \item Международная конференция «Иванниковские чтения». Нижний Новгород 2021.
  \item Международная техническая конференция по открытой СУБД PostgreSQL «PGConf.Russia». Москва. 2021.
  \item Ломоносовские чтения. Научная конференция. Москва. 2022.
  \item IX International Conference «Engineering \& Telecommunication — En\&T-2022». Москва. 2022.
  \item Открытая конференция ИСП РАН им. В.П. Иванникова. Москва. 2022.
\end{enumerate}

{\contribution} Все представленные в диссертации результаты получены лично автором.

\ifnumequal{\value{bibliosel}}{0}
{%%% Встроенная реализация с загрузкой файла через движок bibtex8. (При желании, внутри можно использовать обычные ссылки, наподобие `\cite{vakbib1,vakbib2}`).
    {\publications} Основные результаты по теме диссертации изложены в~5~печатных изданиях,
    2 из которых в периодических научных журналах, индексируемых Web of Science и Scopus,
    3 –– в тезисах докладов [5; 6].
    Зарегистрирована 1 программа для ЭВМ [7].
}%
{%%% Реализация пакетом biblatex через движок biber
    \begin{refsection}[bl-author, bl-registered]
        % Это refsection=1.
        % Процитированные здесь работы:
        %  * подсчитываются, для автоматического составления фразы "Основные результаты ..."
        %  * попадают в авторскую библиографию, при usefootcite==0 и стиле `\insertbiblioauthor` или `\insertbiblioauthorgrouped`
        %  * нумеруются там в зависимости от порядка команд `\printbibliography` в этом разделе.
        %  * при использовании `\insertbiblioauthorgrouped`, порядок команд `\printbibliography` в нём должен быть тем же (см. biblio/biblatex.tex)

        \insertbiblioauthor
        %
        % Невидимый библиографический список для подсчёта количества публикаций:
        \printbibliography[heading=nobibheading, section=1, env=countauthorvak,          keyword=biblioauthorvak]%
        \printbibliography[heading=nobibheading, section=1, env=countauthorwos,          keyword=biblioauthorwos]%
        \printbibliography[heading=nobibheading, section=1, env=countauthorscopus,       keyword=biblioauthorscopus]%
        \printbibliography[heading=nobibheading, section=1, env=countauthorconf,         keyword=biblioauthorconf]%
        \printbibliography[heading=nobibheading, section=1, env=countauthorother,        keyword=biblioauthorother]%
        \printbibliography[heading=nobibheading, section=1, env=countregistered,         keyword=biblioregistered]%
        \printbibliography[heading=nobibheading, section=1, env=countauthorpatent,       keyword=biblioauthorpatent]%
        \printbibliography[heading=nobibheading, section=1, env=countauthorprogram,      keyword=biblioauthorprogram]%
        \printbibliography[heading=nobibheading, section=1, env=countauthor,             keyword=biblioauthor]%
        \printbibliography[heading=nobibheading, section=1, env=countauthorvakscopuswos, filter=vakscopuswos]%
        \printbibliography[heading=nobibheading, section=1, env=countauthorscopuswos,    filter=scopuswos]%
        %
        \nocite{*}%
        %
        {\publications} Основные результаты по теме диссертации изложены в~\arabic{citeauthor}~печатных изданиях,
        \arabic{citeauthorvak} из которых изданы в журналах, рекомендованных ВАК\sloppy%
        \ifnum \value{citeauthorscopuswos}>0%
            , \arabic{citeauthorscopuswos} "--- в~периодических научных журналах, индексируемых Web of~Science и Scopus\sloppy%
        \fi%
        \ifnum \value{citeauthorconf}>0%
            , \arabic{citeauthorconf} "--- в~тезисах докладов.
        \else%
            .
        \fi%
        \ifnum \value{citeregistered}=1%
            \ifnum \value{citeauthorpatent}=1%
                Зарегистрирован \arabic{citeauthorpatent} патент.
            \fi%
            \ifnum \value{citeauthorprogram}=1%
                Зарегистрирована \arabic{citeauthorprogram} программа для ЭВМ.
            \fi%
        \fi%
        \ifnum \value{citeregistered}>1%
            Зарегистрированы\ %
            \ifnum \value{citeauthorpatent}>0%
            \formbytotal{citeauthorpatent}{патент}{}{а}{}\sloppy%
            \ifnum \value{citeauthorprogram}=0 . \else \ и~\fi%
            \fi%
            \ifnum \value{citeauthorprogram}>0%
            \formbytotal{citeauthorprogram}{программ}{а}{ы}{} для ЭВМ.
            \fi%
        \fi%
        % К публикациям, в которых излагаются основные научные результаты диссертации на соискание учёной
        % степени, в рецензируемых изданиях приравниваются патенты на изобретения, патенты (свидетельства) на
        % полезную модель, патенты на промышленный образец, патенты на селекционные достижения, свидетельства
        % на программу для электронных вычислительных машин, базу данных, топологию интегральных микросхем,
        % зарегистрированные в установленном порядке.(в ред. Постановления Правительства РФ от 21.04.2016 N 335)
    \end{refsection}%
    \begin{refsection}[bl-author, bl-registered]
        % Это refsection=2.
        % Процитированные здесь работы:
        %  * попадают в авторскую библиографию, при usefootcite==0 и стиле `\insertbiblioauthorimportant`.
        %  * ни на что не влияют в противном случае
        \nocite{scbib1}%conf
        \nocite{bib1}%conf
        \nocite{confbib1}%conf
        \nocite{confbib2}%conf
        \nocite{bib2}%conf

    \end{refsection}%
        %
        % Всё, что вне этих двух refsection, это refsection=0,
        %  * для диссертации - это нормальные ссылки, попадающие в обычную библиографию
        %  * для автореферата:
        %     * при usefootcite==0, ссылка корректно сработает только для источника из `external.bib`. Для своих работ --- напечатает "[0]" (и даже Warning не вылезет).
        %     * при usefootcite==1, ссылка сработает нормально. В авторской библиографии будут только процитированные в refsection=0 работы.
}
