\chapter{Обзор предметной области}\label{ch:ch1}
\section{Использование программных библиотек}\label{sec:ch1/sec1}
\subsection{Структура программных библиотек}\label{subsec:ch1/sec1/sub1}
Библиотека — это пакет кода, предназначенный для повторного использования во многих программах. Обычно библиотека на языке Си и Cи++ состоит из двух частей:
\begin{itemize}
    \item заголовочный файл, определяющий функциональность, которую библиотека предоставляет (предлагает) программам, которые ее используют;  предварительно скомпилированный двоичный файл, содержащий реализацию этой функциональности, предварительно скомпилированную в собственный код. Некоторые библиотеки могут быть разбиты на несколько файлов и/или иметь несколько заголовочных файлов.
    \item заголовочный файл, определяющий функциональность, которую библиотека предоставляет (предлагает) программам, которые ее используют;предварительно скомпилированный двоичный файл, содержащий реализацию этой функциональности, предварительно скомпилированную в собственный код. Некоторые библиотеки могут быть разбиты на несколько файлов и/или иметь несколько заголовочных файлов.
\end{itemize}

Библиотеки предварительно компилируются по нескольким причинам. Во-первых, поскольку библиотеки редко меняются, их не нужно часто перекомпилировать. Было бы пустой тратой времени перекомпилировать библиотеку каждый раз, когда вы пишете программу, которая ее использует. Во-вторых, поскольку предварительно скомпилированные объекты представляют собой машинный код, люди не могут получить доступ к исходному коду или изменить его, что важно для компаний или людей, которые не хотят делать свой исходный код доступным по причинам интеллектуальной собственности.

Существует два типа библиотек: статические библиотеки и динамические библиотеки.

Статическая библиотека (иногда называемая архивом) состоит из подпрограмм, которые скомпилированы и связаны непосредственно с вашей программой. Когда вы компилируете программу, использующую статическую библиотеку, все функции статической библиотеки, используемые вашей программой, становятся частью вашего исполняемого файла. В Windows статические библиотеки обычно имеют расширение .lib, а в Linux — расширение .a (архив). Одним из преимуществ статических библиотек является то, что вам нужно распространять только исполняемый файл, чтобы пользователи могли запускать вашу программу. Поскольку библиотека становится частью вашей программы, это гарантирует, что с вашей программой всегда будет использоваться правильная версия библиотеки. Кроме того, поскольку статические библиотеки становятся частью вашей программы, вы можете использовать их точно так же, как функции, которые вы написали для своей программы. С другой стороны, поскольку копия библиотеки становится частью каждого исполняемого файла, который ее использует, это может занять много места. Статические библиотеки также не могут быть легко обновлены — обновление библиотеки требует замены всего исполняемого файла.

Динамическая библиотека (также называемая разделяемой библиотекой) состоит из подпрограмм, которые загружаются в ваше приложение во время выполнения. Когда вы компилируете программу, использующую динамическую библиотеку, эта библиотека не становится частью вашего исполняемого файла — она остается отдельным объектом. В Windows динамические библиотеки обычно имеют расширение .dll (библиотека динамической компоновки, библиотека динамической компоновки), а в Linux — расширение .so (общий объект, общий объект). Одним из преимуществ динамических библиотек является то, что многие программы могут совместно использовать одну копию библиотеки, что экономит место. Возможно, самым большим преимуществом является то, что динамическую библиотеку можно обновить до более новой версии, не заменяя все исполняемые файлы, которые ее используют.

Рассмотрим простой пример использования функций libjson-c (самая популярная библиотека для анализа строк в файлах формата JSON ) в библиотеке libstorj на таблице~\cref{tab:exampleLibjson1}.

\begin{table}
    \centering
    \captionsetup{justification=centering}
    \caption{Пример использования функций libjson\-c в библиотеке libstorj}\label{tab:exampleLibjson1}
    \begin{tabular}{|c|l|}
        \hline
        {1} & {struct json\_object *body = json\_object\_new\_object();} \\
        {2} & {json\_object *name = json\_object\_new\_string(bucket\_name);} \\
        {3} & {json\_object\_object\_add(body, "name", name);} \\
        { } & {//строки кода} \\
        {4} & json\_object\_put(body); \\
        \hline
    \end{tabular}
\end{table}

\begin{itemize}
    \item	В первой строке объявляется новый объект «body» типа struct json\_object * и определяется функцией \textbf{json\_object\_new\_object\(\)}.
    \item	{Во второй строке объявляется новый объект «name» типа json\_object и определяется функцией json\_object\_new\_string, которая принимает на вход переменную типа \textit{const char *} с именем \textit{«bucket\_name».}}
    \item	В третей строке к объекту «body» прибавляется объект name c ключевым тегом «name». Здесь функция json\_object\_object\_add принимает 3 аргумента типов: «struct json\_object *», «const char *» и «json\_object *».
    \item В четвертой строке вызывается функция json\_object\_put, аргументом которой является объект «body». Эта функция уменьшает счетчик ссылок на объект «body» и освобождает его, если он достигает нуля.
\end{itemize}

\subsection{Методы тестирования библиотек}\label{subsec:ch1/sec1/sub1}

Параллельно с ростом и развитием индустрии разработки программного обеспечения размер и сложность программных компонентов значительно увеличиваются. Автоматическое тестирование крупных программных компонентов является нетривиальной задачей, в настоящий момент не имеющей полноценного решения, особенно с учетом ограниченных вычислительных мощностей. 

В общем случае, для организации процедуры тестирования программных компонентов, разра-ботчик должен: проанализировать исходный код компонента и доступную документацию, написать тестовые программы с различными наборами входных данных для подлежащих тестированию функций библиотеки, скомпилировать тестовые программы, собрать результат их выполнения, выполнить анализ полученных данных. Этот про-цесс называется модульным тестированием [3] (Unit-test), и это важный подход к уменьшению количества дефектов и повышению качества про-граммного обеспечения. Однако компонент может состоять из сотен функций, десятков определен-ных структур данных, а приложение может ис-пользовать несколько компонент (в том числе раз-деляемых библиотек), поэтому практическая воз-можность вручную протестировать все задейству-емые компоненты как правило отсутствует.


\section{Тестирование программных библиотек}\label{sec:ch1/sec3}


\subsection{Методы тестирования библиотек}\label{subsec:ch1/sec3/sub1}

